\section{Applications in Reinforcement Learning}
\paragraph{Definitions:} We are presented with data in the form of an $i.i.d$ sequence $(\phi_t,\phi'_t,r_t)\in \R^d\times\R^d\times \R$. Let $\gamma \in (0,1)$ be a fixed constant, we define $\Delta_t\eqdef \phi_t\phi_t^\top-\gamma \phi_t\phi_t^{'\top}$, $C_t=\phi_t\phi_t^\top$ and $b_t=\phi_r r_t$. In what follows, $\mu_t>0$ is an \emph{importance} sampling factor whose aim is to correct for mismatch in the (behavior) distribution with which the data was collected and the (target) distribution with respect to which one wants to learn. A factor $\mu_t=1,\,\forall t\geq 0$ will mean that no correction is required\footnote{This is known as the \emph{on-policy} case where the behavior is identical to the target. The general setting where $\mu_t>0$ is known as \emph{off-policy}.}.
\begin{table}
\resizebox{\columnwidth}{!}{
\begin{tabular}{|c|c|c|c|c|}\hline
Algorithm& Update &Stepsize &Rate \\ \hline
TD(0)& $\begin{aligned}\theta_t&=\theta_{t-1}+\alpha_t (b_t -\Delta_t\theta_{t-1})\\ \thh_t&=\frac{1}{t+1}\sum_{s=0}^t \theta_t\end{aligned}$ &$\begin{aligned}\alpha_t&=O(\frac1{t})^\delta,\\ &\delta\in(1/2,1)\end{aligned}$ &$\EE{\norm{\eh_t}}\leq O(\frac{1}{\sqrt{t}})$ for $\delta\ra 1$ \\ \hline
GTD& $\begin{aligned} y_{t}&=y_{t-1}+\beta_t(\mu_t b_t -\mu_t \Delta_t \theta_{t-1}- Q_t y_{t-1})\\ \theta_t &=\theta_{t-1}+\alpha_t (\mu_t A^\top_t y_{t-1})\end{aligned}$ & $\begin{aligned}&\beta_t=\eta \alpha_t,\sum_{t\geq 0}\alpha_t=\infty, \\ &\sum_{t\geq 0}\alpha^2_t<\infty \end{aligned}$ & $e_t\ra 0$ w.p.1 \\\hline
TDC& $\begin{aligned} y_{t}&=y_{t-1}+\beta_t(\mu_t b_t -\mu_t \Delta_t \theta_{t-1}-C_t y_{t-1})\\ \theta_t &=\theta_{t-1}+\alpha_t (\mu_t b_t-\mu_t\Delta_t\theta_{t-1}+ (\Delta^\top_t-C_t)y_{t-1})\end{aligned}$ & $\begin{aligned}&\beta_t,\alpha_t,\frac{\alpha_t}{\beta_t} \ra 0,~\text{as}~t\ra \infty,\\& \sum_{t\geq 0}\alpha_t=\infty, \sum_{t\geq 0}\beta_t=\infty, \\ &\sum_{t\geq 0}\alpha^2_t<\infty, \sum_{t\geq 0}\beta^2_t<\infty \end{aligned}$ & $e_t\ra 0$ w.p.1 \\\hline
GTD& $\begin{aligned} y_{t}&=\Pi_Y \left(y_{t-1}+\alpha_t(\mu_t b_t -\mu_t \Delta_t \theta_{t-1}-Q_t y_{t-1})\right)\\ \theta_t &=\Pi_{\Theta}\left(\theta_{t-1}+\alpha_t (\mu_t A_t^\top y_{t-1})\right)\\ \thh_t&=\frac{\sum_{s=0}^t \alpha_s\theta_s}{\sum_{s=0}^t \alpha_s}\end{aligned}$ & $\alpha_t =O(\frac{1}{\sqrt{t}})$ & $\norm{A_P\eh_t-b}^2 \leq O(\frac{1}{\sqrt{t}})$ w.h.p\\\hline

\end{tabular}
}
\label{tb:rates}
\end{table}
\paragraph{Related Work in RL} The various temporal difference learning algorithms can be written as in \Cref{table:td}. The TD(0) algorithm is an on-policy algorithm ($\mu_t=1$) and is unstable in the general {off-policy} setting. The gradient temporal difference (GTD) algorithm \cite{gtd,gtd2,gtdmp} is stable in off-policy setting as well. Recent work by \cite{korda-prashanth} derive finite time bounds for the TD(0) for two different cases of step-sizes one without PR averaging and other with PR averaging. In the case without PR averaging, \cite{korda-prashanth} consider step-sizes of the form $\frac{c_0c}{c+t}$, where $c_0$ and $c$ are problem dependent positive constants, to dervie a rate of $O(\frac{1}{t})$ for the MSE.  In the case with PR averaging (row $1$ of \Cref{tb:rates}), \cite{korda-prashanth} consider step-size of the from $c_0\left(\frac{c}{c+t}\right)^\delta,\,\delta \in (1/2,1)$ where $c_0, c$ are problem independent positive constants, and show that this choice yeilds a rate of $O(\frac{1}{t})$ for $\delta \ra 1$ and $t\geq t_0$ (where $t_0$ is a problem dependent time instant). A possible limiting feature of the results by \cite{korda-prashanth} is that either the problem knowledge is required or the bounds hold only after a problem dependent time. A centered TD(0) variant is also considered by \cite{korda-prashanth}, however, a projection operator is required to keep the iterates bounded. Also, the bounds by \cite{korda-prashanth} don't have an explicit bias variance decomposition. The works on GTD and TDC by \cite{gtd,gtd2} show only asymptotic convergence w.p.1. Recently, the work by \cite{gtdmp} shows a rate of $\frac{1}{\sqrt{t}}$ for GTD with use of projection operation, and also proposed a variant namely GTD-Mirror-prox based on stochastic mirror-prox ideas which achieves $\frac{1}{\sqrt{t}}$ rate without use of projection. In comparison, we don't use of projections, our bounds hold for all time $t$, and the bounds are dependent on the problem instance only to the extent of choosing a step-size that leads to stable behavior, in which case we show that $O\frac{1}{t}$ rate holds (the scaling factor hidden in the $O$ expression is problem dependent). Further, our results provide explicit bias variance decomposition.
