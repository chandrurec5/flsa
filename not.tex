\section{Notations and Definitions}\label{sec:def}

We denote the sets real and complex number by $\R$ and $\C$ respectively. For $x=a+ib\in \C$, we denote its real and imaginary parts by $\re{x}=a$ and $\im{x}=b$ respectively. We denote $n$-dimensional vector spaces over $\R$ and $\C$ by $\R^{d}$ and $\C^{d}$ respectively. We use $\R^{\dcd}$ and $\C^{\dcd}$ to denote $\dcd$ matrices with real and complex entries respectively. For any $x\in \C$ we denote the modulus of $x$ by $\md{x}=\sqrt{\re{x}^2+\im{x}^2}$. For any $x\in \C^d$We denote complex conjugate of $x$ by $\bar{x}$. We denote the transpose of $x\in \C^{d}, A\in \C^{\dcd}$ by $x^\top$ and $A^\top$ respectively. Further, by $x^*$ denotes the conjugate transpose, i.e., $x^*=\bar{x}^\top$ (the same convention carries over to matrices as well). We will use $\ip{\cdot,\cdot}$ to denote the inner products: $\ip{x,y}=x^* y$. Also, by $A_{ij}$ we denote the entry of matrix $A$ in the $i^{th}$ row and $j^{th}$ column.
The norm of the matrix $A$ in given by $\norm{A}\eqdef \sup_{x\in \C^d:\norm{x}=1} \norm{Ax}$. For a positive real number $B>0$, we denote by $\C^{\dcd}_B=\{A\in \C^{\dcd}\mid \norm{A}\leq B\}$ and  $\R^{\dcd}_B=\{A\in \R^{\dcd}\mid \norm{A}\leq B,\,\forall i,j=1,\ldots,n\}$ the set of complex and real matrices whose norms are bounded $B$.
%We denote by $\C^{\dcd}_B=\{A\in \C^{\dcd}\mid \md{A_{ij}}\leq B,\,\forall i,j=1,\ldots,n\}$ and  $\R^{\dcd}_B=\{A\in \R^{\dcd}\mid \md{A_{ij}}\leq B,\,\forall i,j=1,\ldots,n\}$.
%We use $\P$ to denote the set of probability distribution over $\C^{\dcd}$.
We also denote the set of invertible $\dcd$ complex matrices by $\gln$. We use $A\sim P$ to denote the fact that the random variable $A$ is distributed according to distribution $P$. Let the random matrix $A\sim P$, and $U\in \gln$, we use $P_U$ to denote the distribution of the random matrix $U^{-1}A U$.  We denote the identity matrix in $\C^{\dcd}$ by $\I$.
%We denote the set of distributions over $\C_B^{\dcd}$ by $\P$.

We use $A\succeq 0$ to denote that the
square matrix $A$ is hermitian and positive semidefinite (HPD):
$A = A^*$, $\inf_x x^* A x\ge 0$.
For $A,B$ HPD matrices, $A\succeq B$ holds if $A-B\succeq 0$.
We also use $A\succ B$ similarly to denote that $A-B \succ 0$.
We also use $\preceq$ and $\prec$ analogously. For any $x\in\C^d$x, we denote the general quadratic norm with respect to a $C\succ 0$ by $\norm{x}^2_C\eqdef x^*\, C \,x$.
$\E$ denotes mathematical expectation.

\begin{definition}
Let $P$ be a distribution over $\C^{\dcd}$, then define
\begin{align*}
A_P=\int M dP(M),\quad C_P=\int M^* M dP(M),
\end{align*}
\end{definition}
\begin{definition}\label{def:contract}[Contraction Factors]
Define
\begin{align*}
\rhod{P}&\eqdef {\inf}_{x\in\C^d\colon\norm{x}=1}\ip{x,\left((A_P+A_P^*)-\alpha A_P^* A_P\right)x},\\ \rhos{P}&\eqdef{\inf}_{x\in \C^d\colon\norm{x}=1}\ip{x,\left((A_P+A_P^*)-\alpha C_P\right)x}
\end{align*}
\end{definition}
Note that $\rhod{P}>\rhos{P}$.

\begin{definition}
We call a matrix $A\in \C^{\dcd}$ to be \emph{positive definite} (PD) if $\ip{x,Ax} >0,\,\forall x\neq 0 \in \C^{d}$. We call a matix $A\in \C^{\dcd}$ to be \emph{asymptotically stable} (AS) if all eigenvalues of $A$ have positive real parts. We call a matrix $A\in \C^{\dcd}$ to be \emph{symmetric positive definite} (SPD) is it is symmetric i.e., $A^\top=A$ and PD.
\end{definition}
\begin{definition}\label{distpd}
We call a distribution $P$ to be PD/AS/SPD if $A_P$ is PD/AS/SPD.
\end{definition}
\begin{example}
The matrices $\begin{bmatrix}0.1 &-1\\ 1 & 0.1\end{bmatrix}$, $\begin{bmatrix} 0.1 & 0.1 \\ 0 & 0.1\end{bmatrix}$ and $\begin{bmatrix}0.1 &0 \\ 0 & 0.1\end{bmatrix}$ are examples of AS, PD and SPD respectively.
\end{example}
