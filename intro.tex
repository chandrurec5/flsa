%!TEX root =  flsa.tex
\section{Introduction}\label{sec:intro}
Linear stochastic approximation algorithms (LSAs) 
of the form
%We consider linear stochastic approximation algorithms (LSA) of the general form given by
\begin{align}\label{eq:lsaintro}
\theta_t=\theta_{t-1}+\alpha_t (b_t-A_t \theta_{t-1}),
\end{align}
with $(\alpha_t)_t$ a positive step-size sequence chosen by the user and 
$(b_t,A_t)\in \R^d\times \R^{\dcd}$,  $t\geq 0$, a sequence of identically distributed random variables is widely used in
machine learning, and in particular in reinforcement learning (RL), to compute the solution of the equation 
$\E[b_t] - \E[A_t] \theta = 0$, where $\E$ stands for mathematical expectation.
%where the iterates, $\theta_t\in \R^d$, and step-sizes, $\alpha_t>0$, are algorithmic quantities, and $(b_t,A_t), t\geq 0\in \R^d\times \R^{\dcd}$ is the noisy data sequence presented to the algorithm.
%LSAs are widely applied in machine learning and reinforcement learning (RL), where the aim is to compute an appropriate $\ts\in \R^d$ using only the noisy data sequence and $O(d)$ updates per iteration. 
Some examples of LSAs include the stochastic gradient descent algorithm (SGD) for the problem of linear least-squares estimation  (LSE) \cite{bach,bachaistats}, and the \emph{temporal difference} (TD) class of learning algorithms in RL \cite{sutton,konda-tsitsiklis,KoTsi03:LSA,gtd,gtd2,gtdmp}.
\todoc{konda-tsitsiklis reference resolved to tsitsiklis-van-roy. is this what you want?? I also added KoTsi03:LSA, maybe that's what you wanted.}
%An additional feature in these class of applications is that $A_t$ turns out to be a rank-$1$ matrix and $A_t\theta_{t-1}$ can be obtained in $O(d)$, which is attractive due to the cheap per time step computational requirement.\par

The choice of the step-size sequence $(\alpha_t)_t$ is critical for the performance of LSAs such as \eqref{eq:lsaintro}.
Informally speaking, smaller step-sizes are better for noise rejection and larger step-sizes lead to faster forgetting of initial conditions (smaller bias). At the same time, step-sizes that are too large might result in instability of \eqref{eq:lsaintro} even when $(A_t)_t$ has favourable properties. 
A useful choice has been the diminishing step-sizes \cite{gtd2,gtdmp,konda-tsitsiklis}, where $\alpha_t\ra 0$ such that $\sum_{t\geq 0} \alpha_t=\infty$. Here, $\alpha_t\to0$ circumvents the need for guessing the magnitude of step-sizes that stabilize the updates, while the second condition ensures that initial conditions are forgotten. 
An alternate idea, which we call LSA with constant step-size and Polyak-Ruppert averaging (LSA with CS-PR, in short), is to run \eqref{eq:lsaintro} by choosing $\alpha_t=\alpha>0$ $\forall t\geq 0$ with some $\alpha>0$, and output the average $\thh_t\eqdef\frac{1}{t+1}\sum_{i=0}^t \theta_i$. Thus, in LSA with CS-PR, $\theta_t$ is an internal variable and $\thh_t$ is the output of the algorithm. The idea is that the constant step-size leads to faster forgetting of initial conditions, while the averaging on the top
reduces noise.
This idea goes back to  \citet{ruppert} and \citet{polyak-judisky} who considered it in the context of stochastic approximation that LSA is a special case of. 
%Intuitively, the CS-PR technique can be seen to address the learning vs noise rejection trade-off, where the constant step-size takes care of the learning  and the averaging of the iterates takes care of the noise rejection.
\paragraph{Motivation and Contribution:} Recently, \citet{bach} considered stochastic gradient descent (SGD)\footnote{SGD is an LSA of the form in \eqref{eq:lsaintro}.} with CS-PR for LSE and \iid sampling. They showed that one can calculate a constant step-size from only a bound on the magnitude of the noisy data so that the leading term as $t\to\infty$
 in the mean-squared prediction error after $t$ updates is at most $\frac{C}{t}$ with a constant $C>0$ that depends \emph{only} on the bound on the data, and is in particular independent of the eigenspectrum of $\E[A_t]$, a property which is not shared by other step-size tunings and variations of the basic SGD method.%
 \footnote{See \cref{sec:related} for further discussion of the nature of these results.}
\todoc{Dimension?}

In this paper, we study LSAs with CS-PR (thereby extending prior work by \citet{bach} from SGD to general LSAs) \todoc{Do our results imply theirs? Do we simplify improve their bounds, while extending the scope? Even if the answer is no, we need to add remarks on these questions.}
in an effort to understand the effectiveness of the CS-PR technique beyond SGD.
Of particular interest is whether a similar result to that  \citet{bach} holds
for the TD class of LSA algorithms used in RL.
For simplicity, we still consider the \iid case. Our restrictions on the common distribution is that the ``noise variance'' should be bounded (as we consider squared errors), and that the matrix $\E[A_t]$ must be Hurwitz, i.e., all its eigenvalues have positive real parts. %This latter assumption can be shown to be necessary for the stability of the LSA iterations. \todoc{Add citation. Or remove/comment out the sentence.}
One setting that fits our assumption is \emph{policy evaluation} \cite{dann} using linear value function approximation from experience replay \cite{lin} in a batch setting \cite{lange} in RL using the TD class of algorithms \cite{sutton,konda-tsitsiklis,gtd,gtd2,gtdmp}. 

Our \textbf{main contributions} are as follows:
%While asymptotically diminishing step-sizes are common in RL literature \cite{ilstd,gtdmp,korda-prashanth}, the constant step-size choice has also been used in the past \cite{gtd2}.
\begin{itemize}[leftmargin=*]%, before = \leavevmode\vspace{-\baselineskip}]
\item \textbf{Finite-time Instance Dependent Bounds}: For a given $P$, we  measure the performance of a given LSA with CS-PR in terms of the mean square error (MSE) given by $\EEP{\normsm{\thh_t-\ts}^2}$. 
For the first time in the literature,
we show that (under our stated assumptions) there exists an $\alpha_P>0$ such that 
for any $\alpha\in (0,\alpha_P)$,
the MSE %of a given LSA with CS-PR
is at most $\frac{C_{P,\alpha}}{t}+\frac{C_{P',\alpha}}{t^2}$ with some positive constants $C_{P,\alpha},C_{P',\alpha}$ that we explicitly compute from $P$.
%The MSE can further be broken down to a \emph{bias} term (arising due to the initial condition $\theta_0$) that decays at a rate $O(\frac{1}{t^2})$ and a \emph{variance} term (due to the noise) that decays at at rate $O(\frac{1}{t})$.
\item \textbf{Uniform Bounds}:
It is of major interest to know whether for a given class $\P$ of distributions 
one can choose some step-size $\alpha$ 
such that $C_{P,\alpha}$ from above is uniformly bounded (i.e., replicating the result of \citet{bach}).%
\footnote{Of course, the term $C_{P',\alpha}/t^2$ needs to be controlled, as well. Just like \citet{bach}, here we focus on $C_{P,\alpha}$, which is justified if one considers the MSE as $t\to\infty$. Further justification is that we actually find a negative result. See above.}
%It is of interest to know that for a given LSA with CS-PR and a given class $\P$ of distributions, whether it is possible to ensure uniform performance $\forall P\in \P$. 
We show via an example that in general this is not possible.
%that there is a class $\P$ that does not `admit' a constant step-size $\alpha_{\P}$ that guarantees uniform performance $\forall P\in \P$. 
In particular, the example applies to RL, hence, we get a negative result for RL, which states that from only bounds on the data one cannot choose a step-size $\alpha$ to guarantee that $C_{P,\alpha}$ of CS-PR is uniformly bounded over $\P$.
We also define a subclass  $\P_{\text{SPD},B}$ of problems, related to SGD for LSE, that does `admit' a uniform constant step-size, thereby recovering a part of the result by \citet{bach}.
Our results in particular shed light on the precise structural assumptions that are needed 
to achieve a uniform bound for CS-PR. 
For further details, see \cref{sec:related}.
\item \textbf{Automatic Step-Size}: 
The above negative result implies that in RL one needs to choose the constant step-size based on properties of the instance $P$ to avoid the explosion of the MSE.
To circumvent this, we propose a natural step-size tuning method to guarantee instance-dependent boundedness.
We experimentally evaluate the proposed method and find that it is indeed able to achieve its goal on a set of synthetic examples
where no constant step-size is available to prevent exploding MSE. %Further, it is empirically demonstrated that the method is able to find a step-size, which is close to the theoretically recommended stepsize which results in the fastest convergence. \todoc{True??}
\end{itemize}
%\paragraph{Implications:} 
In addition to TD($0$), our results directly can be applied to other \emph{off-policy} TD algorithms such as GTD/GTD2 with CS-PR. \todoc{How about TD($\lambda$)? Will we discuss this somewhere? Maybe in the organization section this can be mentioned.}
%In the case of TDC only asymptotic bounds were known \cite{gtd2} and our results are a first step
%\todoc{Why only a step? What is missing?} towards proving finite time bounds.
In particular, our results show that the GTD class of algorithms guarantee a $O(\frac{1}{t})$ rate for MSE (without use of projections), improving on a previous result by \citet{gtdmp} that guaranteed a $O(\frac{1}{\sqrt{t}})$ rate for this class for the projected version of the algorithm. \todoc{And projections are problematic on their own. Will we discuss this somewhere. Or should the projection issue be a footnote here?}
\todoc{How about Prashanth's paper? Should we mention it somewhere? The reviewers will push back if we don't.}
%\paragraph{Organization:} The paper is organized as follows \todoc{Do not forget to fill this in.}
