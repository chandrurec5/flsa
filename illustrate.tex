\subsection{Illustrative Examples}
In this section, we present some illustrative examples and make some useful remarks about the same.
\begin{example}\label{ex:pdas}
Matrices $A_1=\begin{bmatrix} 2 &1 \\-1 &0\end{bmatrix}$$A_2=\begin{bmatrix}1 & -1\\ 1 &0 \end{bmatrix}$ and $A_3=\begin{bmatrix} 1 & 10-\epsilon \\ 0 &1 \end{bmatrix},0<\epsilon<0.1$ are not positive definite. However choosing $U_1=\begin{bmatrix} \frac{1}{\sqrt{2}}& -\frac{1}{\sqrt{2}}\\ -\frac{1}{\sqrt{2}} &\frac{1}{\sqrt{2}} \end{bmatrix}$, $U_2=\begin{bmatrix} \frac{\sqrt{2}}{1-i\sqrt{3}} & \frac{\sqrt{2}}{1+i\sqrt{3}} \\ -\frac{1}{\sqrt{2}} &\frac{1}{\sqrt{2}}\end{bmatrix}$ and $U_3=\begin{bmatrix} 1 & 0\\ 0 &\frac{1}{10} \end{bmatrix}$, we can see that matrices $J_i=U_i^{-1}A_i U_i,i=1,2,3$ are positive definite.
\end{example}
The above example \Cref{ex:pdas} illustrates the use of appropriate $U_i,i=1,2,3$ matrices to transform Hurwitz problems into positive definite problems. In particular, problem $P^2$ involves the use of a complex matrix $U_2$.
\begin{example}\label{ex:geo}
Consider the one dimensional case where $A_t=\rho$ for some $0<\rho<1$, $b_t$ a bounded $i.i.d$ zero mean random variable with variance $\sigma^2$, and $\ts=0$. We know that $e_t=(1-\alpha\rho)^t e_0+\sum_{s=1}^t (1-\alpha \rho)^{t-s}\alpha b_s$ and the MSE is given by
\begin{align*}
\EE{\norm{\thh_t}^2}&=\frac{1}{(t+1)^2}\left( (\alpha\rho)^{-1}(1-\alpha\rho)^{t+1}) \theta_0+ \sum_{s=1}^t (\alpha\rho)^{-1}(1-\alpha \rho)^{t-s+1} \alpha b_s\right)^2\\
&\leq (\alpha\rho)^{-2}\left(\frac{\norm{\theta_0}^2}{t^2}+\frac{\alpha^2\sigma^2}{t}\right)
\end{align*}
\end{example}
The above example helps to interpret the error dynamics of \Cref{eq:lsa} in terms of sum of a geometric series with common factor $\alpha\rho$.
\begin{comment}
\begin{example}
Consider deterministic problems $P^1$, $P^2$ and $P^3$ with $A_{P^1}=\begin{bmatrix} 10 &100\\ 0 &10\end{bmatrix}$, $A_{P^2}=\begin{bmatrix} 10 &9.99\\ 0 &10\end{bmatrix}$, $A_{P^3}=\begin{bmatrix} 10 &0 \\ 0 &10\end{bmatrix}$, $\sigma^2_{A_{P^i}}=\sigma_{b_{P^i}}=0,\,i=1,2,3$. Note that $P^1$, $P^2$ and $P^3$ are respectively H, PD, SPD respectively.
\end{example}
\FloatBarrier
\begin{figure}[htp]
\resizebox{\columnwidth}{!}{
\begin{tabular}{ccc}
\begin{tikzpicture}[scale=0.3,font=\Large]
    \begin{axis}[
        xlabel=$t$,
         ylabel=$\norm{\thh_t-\ts}^2$, legend style={at={(0.5,-0.1)},anchor=north}
]
    \addplot[only marks,mark=square,red] plot file {./exp/as};
    \end{axis}
    \end{tikzpicture}
&
\begin{tikzpicture}[scale=0.3,font=\Large,]
\begin{axis}[
xlabel=$t$,
ylabel=$\norm{\thh_t-\ts}^2$, legend style={at={(0.5,-0.1)},anchor=north}
]
\addplot[only marks,mark=square,red] plot file {./exp/pd};
\end{axis}
\end{tikzpicture}

&
\begin{tikzpicture}[scale=0.3,font=\Large]
\begin{axis}[
xlabel=$t$,
ylabel=$\norm{\thh_t-\ts}^2$, legend style={at={(0.5,-0.1)},anchor=north}
]
\addplot[only marks,mark=square,red] plot file {./exp/spd};
\end{axis}
\end{tikzpicture}
\end{tabular}
}
\end{figure}

\end{comment}
