\section{Uniform stepsizes}
If $\P$ is weakly admissible, then one can choose a stepsize $\alpha_{\P}>0$ solely based on the knowledge of $\P$ and
conclude that no matter the joint distribution $P$ of $(b_t,A_t)$, as long as the distribution of $P$ is an element of $\P$, the conclusions of \Cref{th:rate} hold. Further, if $\P$ is admissible then the error bound stated in  \cref{maintheorem} becomes independent of the instance. Hence, an interesting question to investigate is whether a given set $\P$ is (weakly) admissible. We consider some reasonable classes of distributions and test whether or not they are (weakly) admissible.
\begin{definition}\label{def:pclass}
\begin{itemize}[leftmargin=*, before = \leavevmode\vspace{-\baselineskip}]
\item For $B>0$, let $\P_B$ be the set of distributions over $\C_B^{d}\times \C_B^{\dcd}$ where for every $P\in \P_B$ is positive definite.
\item  $\P_{\text{PSD},B}$ be such that  $\P^M_{\text{PSD},B}=\{ P^M \in \P^M_B \,:\,  \supp(P^M)\subset \text{PSD} \}\,$ , where PSD denotes the set of positive semi-definite matrices.
\item  $\P_{\text{PD},B}$ be such that  $\P^M_{\text{SD},B}=\{ P^M \in \P^M_B \,:\,  \supp(P^M)\subset \text{PD} \}\,$ , where PD denotes the set of positive definite matrices.
\end{itemize}
\end{definition}
Note that all distributions are a subset of $\P_B$ and in most applications the knowledge of a suitable $B>0$ is available,and by appropriately normalizing the vectors and matrices involved, we may as well assume that $B=1$ without any loss of generality.
\begin{proposition}\label{notwad}
The set $\P_B$ is not weakly admissible.
\end{proposition}
\begin{proposition}
For any $B>0$, the set $\P_{\text{PSD},B}$ is weakly admissible
and in particular any $0<\alpha < 2/B$ witnesses this.
\end{proposition}
\begin{proposition}
For any $B>0$, the set $\P_{\text{PSD},B}$ is not admissible.
\end{proposition}

\begin{comment}
\begin{proof}
Fix an arbitrary $\alpha>0$. We show that there exists $P\in \P$ such that $\rho_\alpha(P)<0$.
For $\epsilon \in (0,1/2)$ let $P=(P^V,P^M)$ be the distribution such that $P^M$ is supported on $\{-I,I\}$ and takes on the value of $I$ with probability $1/2+\epsilon$. Then $A_P = 2\epsilon I \succ 0$, hence $P\in \P_1$. Further, $Q_P = I$.
Hence, $\rhos{P} = 4\epsilon-\alpha$. Hence, if $\epsilon<\alpha/4$, $\rho_\alpha(P)<0$.
\end{proof}
\end{comment}

\begin{comment}
\begin{proof}
Take any $P\in \P_{\text{PSD}}$ and let $H\sim P$.
Consider the SVD of $H$: $H = U \Lambda U^\top$ where $U$ is orthonormal and $\Lambda$ is diagonal with
nonnegative elements. Note that $\Lambda \preceq B\, \I$ and thus $\Lambda^2 \preceq B \Lambda$.
Then for any $x\in \R^d$, $x^\top H^\top H x = x^\top U \Lambda^2 U^\top x \le B x^\top U \Lambda U^\top x = B x^\top H x$.
Taking expectations we find that $x^\top Q_P x \le B x^\top H_P x$.
Hence, $\rho_\alpha(P) = 2 x^\top H_P x - \alpha x^\top Q_P x \ge (2- \alpha B ) \,x^\top H_P x $.
Thus, for any $\alpha<2/B$, $\rho_\alpha(P)>0$.
\end{proof}
\end{comment}
\todoc[inline]{So this altogether is weaker than what Bach founds. They prove strong admissibility.
I am guessing that for this they use that the loss is special and also that the noise is special.
This would be a good place to somehow incorporate this with a (short?) proof.
}
Thus \Cref{notwad} positive definiteness of $P\in \P_B$ alone is not sufficient for weak admissibility. However, when we impose additional structure in the data in the form of positive semidefiniteness,  we then observe weak admissibility in \Cref{wad}, and at the same time, it is not enough to guarantee admissiblity in \Cref{notad}. Finally, when we impose strucutre in the form of positive definiteness we observe admissibility.
\FloatBarrier
\begin{table}[H]
\begin{tabular}{|c|c|c|c|}\hline
Problem Class & Weak Admissible & Admissible &Example \\ \hline
$\P_B$ &\ding{53} &\ding{53} & RL\\\hline
$\P_{\text{PSD},B}$ &\ding{51} &\ding{53} & SGD for Linear Regression\\\hline
$\P_{\text{PD},B}$ &\ding{51} &\ding{51} & SGD for Linear Regression with regularization\\\hline
\end{tabular}
\end{table}
